\section{Abbildungen}
\label{sec:Abbildungen}

Eine Abbildung $f$ ordnet \emph{jedem} Element der Menge $A$ \emph{genau ein} Element der Menge $B$ zu.
	
	\begin{align*}
	f:\quad A&\rightarrow B  \\
	x&\mapsto y=f(x)	
	\end{align*}
$A$: Definitionsbereich von $f$ \\
$B$: Wertebereich von $f$

\[
\int_1^2 f(x) dx
\]

\section{Gleichheit von zwei Abbildungen}
\label{sec:GleichheitVonZweiAbbildungen}

Zwei Abbildungen $f$, $g$ heissen \emph{gleich} genau dann, wenn ihre \emph{Definitions-} und \emph{Wertebereiche �bereinstimmen}, und wenn f�r jedes $x\in A$ gilt:
\[
	f(x)=g(x)
\]


\section{Lineare Abbildungen von $\R ^n \rightarrow \R ^m$}
\label{sec:LineareAbbildungenVonRNRM}

\[
\underbrace{\left( \begin{array}{c} y_1 \\ y_2 \\ \vdots \\ y_n \end{array} \right)}_{y} =
f\left( \begin{array}{c} x_1 \\ x_2 \\ \vdots \\ x_n \end{array} \right) = 
\underbrace{\left( \begin{array}{cccc} a_{11} & a_{12} & \hdots & a_{1n}\\ 
																			 a_{21} & a_{22} & \hdots & a_{2n} \\
																			 \vdots & \vdots & \ddots & \vdots \\ 
																			 a_{m1} & a_{m2} & \hdots & a_{mn} \end{array}\right)}_{A_f} \cdot 
\underbrace{\left( \begin{array}{c} x_1 \\ x_2 \\ \vdots \\ x_n \end{array} \right)}_{x}
\]
Die Matrix $A_f$ heisst die \emph{Standarddarstellungsmatrix}.


\section{Aufstellen der Standarddarstellungsmatrix einer Funktion $f$}
\label{sec:AufstellenDerStandarddarstellungsmatrixEinerFunktionF}

Die Standarddarstellungsmatrix einer Funktion $f: \R ^n \rightarrow \R ^m$ erh�llt man, indem man die Funktion $f$ auf die Standardeinheitsvektoren von $\R ^n$ ($e_1, e_2,\ldots, e_n$) anwendet.
\[
A_f = \left(f(e_1) \quad f(e_2) \quad \cdots \quad f(e_n)\right) =
\underbrace{\left(f \left( \begin{array}{c} 1 \\ 
																0  \\
																\vdots  \\ 
																0  \end{array}\right) \quad
f \left( \begin{array}{c} 0 \\ 
													1  \\
													\vdots  \\ 
													0  \end{array}\right) \quad
\cdots \quad
f \left( \begin{array}{c} 0 \\ 
													0  \\
													\vdots  \\ 
													1  \end{array}\right) \quad
\right)}_{(n\times m)-Matrix}
\]


\section{Komposition von linearen Abbildungen}
\label{sec:KompositionVonLinearenAbbildungen}

Das hintereinanderschalten von linearen Abbildungen
	\begin{align*}
	\R ^p \stackrel{g}{\longrightarrow} \R ^n \stackrel{f}{\longrightarrow} \R ^m \\
	x \mapsto y=g(x) \mapsto z=f(y)=f(g(x))	
	\end{align*}
mit passendem Werte- und Definitionsbereich wird als die \emph{Komposition} $f\circ g$ bezeichnet.

	\begin{align*}
	(f\circ g):\quad \R ^p &\rightarrow \R ^m  \\
	x&\mapsto (f\circ g)(x) := f(g(x))	
	\end{align*}
Die Standarddarstellungsmatrix einer komponierten Funktion $f\circ g$ erh�llt man aus der Matrix-Multiplikation der einzelnen Standarddarstellungsmatrizen $A_f$, $A_g$.
\[
A_{f\circ g} = A_f \cdot A_g
\]
Die Reihenfolge in der lineare Abbildungen kompmoniert werden spielt eine Rolle. Allg: $f\circ g \neq g\circ f$


\section{Eigenschaften von linearen Abbildungen}
\label{sec:EigenschaftenVonLinearenAbbildungen}

Eine Abbildung $f: \R ^n \rightarrow \R ^m$ ist linear genau dann, wenn sie folgende Eigenschaften erf�llt:

\[
f(x+y) = f(x) + f(y) \quad (\forall x, y \in \R ^n )
\]
\[
f(c\cdot x) = c\cdot f(x) \quad (\forall x \in \R ^n, \forall c \in \R)
\]


\section{Kern und Bild einer Abbildung}
\label{sec:KernUndBildEinerAbbildung}

\subsection{Kern einer Abbildung}
\label{sec:KernEinerAbbildung}
Die Teilmenge der Definitionsbereichs $\R ^n$ der linearen Abbildung
$f: \R ^n \rightarrow \R ^m$, welcher auf ${O} \in \R ^m$ (Nullvektor) abgebildet wird, heisst der \emph{Kern} von $f$.
\[
ker \, f =\left\{ u \in \R ^n, f(u)=O \in \R ^m\right\}
\]
Eine lineare Abbildung ist injektiv falls $ker f = \left\{ 0 \right\}$

\subsection{Bild einer Abbildung}
\label{sec:BildEinerAbbildung}
Die Teilmenge des Wertebereichs $\R ^m$, die von $f$ als Bildelemente beansprucht werden, heisst das \emph{Bild} von $f$.
\[
im \, f = \left\{ f(u) \in \R ^m, u \in \R ^n \right\}
\]

\section{Umkehrabbildungen von linearen Abbildungen}
\label{sec:UmkehrabbildungenVonLinearenAbbildungen}

Die Umkehrabbildung einer linearen Abbildung erh�llt man, indem man die Standarddarstellungsmatrix $A_f$ invertiert.
\[
A_{f^{-1}} = A_f^{-1}
\]
$A_f$: ($n \times n$)-Matrix \\
$\det A_f \neq 0$

\section{Spiegelungen in der Ebene}
\label{sec:SpiegelungenInDerEbene}
Mithilfe der folgenden Standarddarstellungsmatrizen kann man einen Vektor $w$ in der Ebene $\R ^2$ spiegeln.
\[
f(x) = A_f \cdot w
\]

\subsection{Spiegelung an der x-Achse}
\label{sec:SpiegelungAnDerXAchse}
\[
A_f = \left( \begin{array}{cc} 1 & 0 \\ 
															 0 & -1
			\end{array}\right)
\]

\subsection{Spiegelung an der y-Achse}
\label{sec:SpiegelungAnDerYAchse}
\[
A_f = \left( \begin{array}{cc} -1 & 0 \\ 
															 0 & 1
			\end{array}\right)
\]

\subsection{Spiegelung an einer Geraden (durch $O$)}
\label{sec:SpiegelungAnEinerGeradenDurchO}
Einen Vektor $w$ spiegelt man an einer Geraden $g = \left\{ r\cdot \left( \begin{array}{c} v_x \\v_y\end{array}\right),\, r \in \R \right\}$ mit folgender Standarddarstellungsmatrix:
\[
A_f = \frac{1}{v_x^2+v_y^2}\left( \begin{array}{cc} 
															v_x^2 - v_y^2   & 2v_xv_y \\ 
															2v_xv_y         & -v_x^2 + v_y^2
														\end{array}\right)
\]


\section{Reflexionen im Raum}
\label{sec:ReflexionenImRaum}
Mithilfe der folgenden Standarddarstellungsmatrizen kann man einen Vektor $w$ im Raum $\R ^3$ spiegeln.
\[
f(x) = A_f \cdot w
\]

\subsection{Spiegelung an der x-y-Ebene}
\label{sec:SpiegelungAnDerXYEbene}
\[
A_f = \left( \begin{array}{ccc} 1 & 0 & 0 \\ 
															  0 & 1 & 0 \\
															  0 & 0 & -1 \\
			\end{array}\right)
\]

\subsection{Spiegelung an der x-z-Ebene}
\label{sec:SpiegelungAnDerXZEbene}
\[
A_f = \left( \begin{array}{ccc} 1 & 0 & 0 \\ 
															  0 & -1 & 0 \\
															  0 & 0 & 1 \\
			\end{array}\right)
\]

\subsection{Spiegelung an der y-z-Ebene}
\label{sec:SpiegelungAnDerYZEbene}
\[
A_f = \left( \begin{array}{ccc} -1 & 0 & 0 \\ 
															  0 & 1 & 0 \\
															  0 & 0 & 1 \\
			\end{array}\right)
\]

\subsection{Spiegelung an einer beliebigen Ebene (durch $O$)}
\label{sec:SpiegelungAnEinerBeliebigenEbeneDurchO}
Um einen Vektor $w$ an einer gegebenen Ebene $E$ zu spiegeln verwendet man folgende Standarddarstellungsmatrix:
\[
A_f = \left( \begin{array}{ccc} 1 - 2n_x^2 & -2n_xn_y & -2n_xn_z \\ 
															  -2n_xn_y & 1 - 2n_y^2 & -2n_yn_z \\
															  -2n_xn_z & -2n_yn_z & 1 - 2n_z^2 \\
			\end{array}\right)
\]
$n$: Einheitsnormalenvektor der Ebene $E$.

\subsubsection{Einheitsnormalenvektor einer Ebene}
\label{sec:EinheitsnormalenvektorEinerEbene}
Gegeben sei die Ebene $E$
\[
E = \left\{ a\cdot \left( \begin{array}{c} u_x \\u_y \\u_z \end{array}\right) + b\cdot \left( \begin{array}{c} v_x \\v_y \\v_z\end{array}\right),\, a,b \in \R \right\}
\]
oder
\[
Ax + By + Cz = 0
\]
Den Einheitsnormalenvektor zur Ebene $E$ erh�llt man aus:
$n = \frac{u\times v}{\left|u\times v\right|}$ oder $n = \frac{\left( \begin{smallmatrix} A \\B \\C \end{smallmatrix}\right)}{\left|\left( \begin{smallmatrix} A \\B \\C \end{smallmatrix}\right)\right|}$


\section{Normalprojektion auf eine Gerade durch $O$ in $\R ^2$}
\label{sec:NormalprojektionAufEineGeradeDurchOInR2}
Um einen Vektor $w$ auf eine gegebene Gerade
$g = \left\{r \cdot \left( \begin{array}{c} u_x \\ 
															  						u_y 
			\end{array}\right), \, r \in \R \right\}$
zu projizieren verwendet man folgende Standarddarstellungsmatrix.
\[
A_f = \frac{1}{u_x^2+u_y^2}\left( \begin{array}{cc} 
														u_x^2	& u_xu_y \\ 
														u_xu_y & u_y^2
														\end{array}\right)
\]


\section{Normalprojektion auf eine Ebene durch $O$ in $\R ^3$}
\label{sec:NormalprojektionAufEineEbeneDurchOInR3}

Um einen Vektor $w$ auf eine gegebene Ebene $E$ (siehe: \ref{sec:EinheitsnormalenvektorEinerEbene}) zu
projizieren verwendet man folgende Standarddarstellungsmatrix.

\[
A_f = \left( \begin{array}{ccc} 1-n_x^2 & -n_xn_y & -n_xn_z \\ 
															  -n_xn_y & 1-n_y		& -n_yn_z \\
															  -n_xn_z & -n_yn_z & 1-n_z^2
			\end{array}\right)
\]
$n$: Einheitsnormalenvektor auf Ebene $E$. Siehe \ref{sec:SpiegelungAnEinerBeliebigenEbeneDurchO}


\section{Drehungen in der Ebene}
\label{sec:DrehungenInDerEbene}
Eine Drehung in $\R^2$ um den Nullpunkt mit dem winkel $\varphi$ l�sst sich als eine lineare Abbildung $f$ mit der Standarddarstellungsmatrix
\[
A_f = \left( \begin{array}{cc} \cos \varphi & -\sin \varphi\\ 
															  \sin \varphi & \cos \varphi
			\end{array}\right)
\]
darstellen.


\section{Drehungen im Raum}
\label{sec:DrehungenImRaum}
Drehungen im dreidimensionalen Raum haben eine gerichtete Drehachse. Zur bestimmung der positiven Drehrichtung verwendet man die "`Rechte-Hand Regel"'.

\subsection{Drehung um die positive x-Achse}
\label{sec:DrehungUmDiePositiveXAchse}
\[
A_f = \left( \begin{array}{ccc} 1 & 0 & 0 \\ 
															  0 & \cos \varphi & -\sin \varphi\\ 
															  0 & \sin \varphi & \cos \varphi
			\end{array}\right)
\]

\subsection{Drehung um die positive y-Achse}
\label{sec:DrehungUmDiePositiveYAchse}
\[
A_f = \left( \begin{array}{ccc} \cos \varphi & 0 & \sin \varphi\\ 
																	0					&	1	&	0 \\
															  -\sin \varphi &0 &  \cos \varphi
			\end{array}\right)
\]

\subsection{Drehung um die positive z-Achse}
\label{sec:DrehungUmDiePositiveZAchse}
\[
A_f = \left( \begin{array}{ccc} \cos \varphi & -\sin \varphi & 0\\ 
															   \sin \varphi & \cos \varphi & 0 \\
															   0				&		0					&			1
			\end{array}\right)
\]



\section{Eigenwerte und Eigenvektoren von liearen Abbildungen $\R^n\rightarrow \R^n$}
\label{sec:EigenverteUndEigenvektorenVonLiearenAbbildungenRNRN}

Den \emph{Skalar} $\lambda \in \R$, f�r den die Gleichung
\[
f(w)=\lambda \cdot w
\]
nicht-triviale L�sungsvektoren $w \in \R^n$ besitzt nennt man \emph{Eigenwert} von $f$.\\
Die nicht-trivialen \emph{L�sungsvektoren} $w \in \R^n$ zu einem gegebenen Eigenwert $\lambda$ heissen \emph{Eigenvektoren} von $f$ zum Eigenwert $\lambda$.

\subsection{Bestimmung der Eigenwerte}
\label{sec:BestimmungDerEigenwerte}
Durch l�sen des Gleichungssystems
\[
(A_f - \lambda \cdot I_n) = O
\]
ergeben sich die Eigenwerte $\lambda_i$.

Feststellungen:
\begin{enumerate}
	\item $\det(A_f - \lambda \cdot I_n)$ ist ein Polynom n-ten Grades in der Variablen $\lambda$ (\emph{charakteristische Polynom} von $f$)
	
	\item Das Poynom $\det(A_f - \lambda \cdot I_n)$ hat h�chstens $n$ reelle Nullstellen
	
	\item Diese \emph{Nullstellen} sind \emph{genau die Eigenwerte} von $f$
\end{enumerate}


\subsection{Bestimmung der Eigenvektoren zu einem Eigenwert von $\lambda$}
\label{sec:BestimmungDerEigenvektorenZuEinemEigenwertVonLambda}

Den Eigenvektor $w_i$ zu einem Eigenwert $\lambda_i$ erh�llt man indem man das Gleichungssystem
\[
(A_f - \lambda_i \cdot I_n) \cdot w_i = O
\]
l�st. Zu einem Eigenwert gibt es immer eine mindestens einparametrige Schar von Eigenvektoren ($\infty$-viele L�sungen). Die Nulll�sung wird nicht als L�sung angesehen.
\\
Falls $w \in \R^n$ ein Eigenvektor zum Eigenwert $\lambda$ ist, so ist f�r jeden Skalar $ \alpha \in \R$ auch $\alpha \cdot w$ ein Eigenvektor zu $\lambda$.


\section{Eigenraum zu einem Eigenwert}
\label{sec:EigenraumZuEinemEigenwert}

Die Menge aller Eigenvektoren zu einem Eigenwert $\lambda$ von $f$, erg�nzt um den Nullvektor, nennt man den \emph{Eigenraum} von $\lambda$.