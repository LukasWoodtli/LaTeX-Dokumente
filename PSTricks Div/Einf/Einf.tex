\documentclass[a4paper,12pt]{scrartcl}

\usepackage[ngerman]{babel}
\usepackage[T1]{fontenc}
\usepackage[latin9]{inputenc}
\usepackage{textcomp}
\usepackage{float}

\usepackage{pstricks}
\usepackage{pst-all}
\usepackage{pstricks-add, pst-plot, pst-node}
\usepackage{pst-pdf}
\title{Titel des Dokuments}
\author{Lukas Woodtli}
\date{\copyright \today}

%
\begin{document}
%\maketitle
%\tableofcontents
\subsection{blab}
	
texttext12, text, text12, text, text12, text, text12, text, text12, text, text12, text, text12, text, text12, text, text12, text, text12, text, text12, text, text12, text, text12, text, text12, text, text12, text, text12, text, 

\begin{figure}[hbt]
\begin{center}
\psset{unit=1cm}
\begin{pspicture}(0, 0)(4, -5)
	%\psgrid
	\pscircle(0, 0){0.1}
	\psline[linewidth=3pt, linecolor=gray](0, 0)(3, 0)
	\psline(0, -1)(3, -1)
	{
	\psline(0, -2)(3, -2)
	\psset{linestyle=dashed}
	\psline(0, -3)(3, -3)
	\psline(0, -4)(3, -4)
	}
	\psline(0, -5)(3, -5)
\end{pspicture}
\caption{testtest}
\end{center}
\end{figure}
text text12, text, text12, text, text12, text, text12, text, text12, text, text12, text, text12, text, text12, text, text12, text, text12, text, text12, text, text12, text, text12, text, text12, text, text12, text, text12, text, 

\section{bla}
text12, text, text12, text, text12, text, text12, text, text12, text, text12, text, text12, text, text12, text, text12, text, text12, text, text12, text, text12, text, text12, text, text12, text, text12, text, text12, text, text12, text, text12, text, text12, text, text12, text, text12, text, text12, text, text12, text, text12, text, 

\begin{figure}[!hbt]
\begin{center}
\begin{pspicture}(-1, 1)(2, 2)
	\psset{arrows=->, unit=3cm, arcangle=20, labelsep=2.5pt, subgriddiv=10, shortput=nab}
	\psgrid
	\pnode(-0.5, 0){0}
	
	\cnodeput(0, 0){q0}{$q_0$}
	\cnodeput(1, 0){q1}{$q_1$}
	\cnodeput[doubleline=true](2, 0){q2}{$q_2$}
	{ 
		\small %Beschriftungen der Verbindungen sollen kleiner gesetzt werden
		\ncarc{0}{q0} %. Anfangszustand
		\ncarc{q0}{q1}^{$0$}
		\ncarc{q1}{q2}^{$0$}
		\ncarc{q1}{q0}^{$1$}
		%die �Loops� � q0-q0-Verbindungen etc.
		\nccircle[angle=180]{q0}{0.4cm}_{$1$} %Der soll nach unten gehen
		\nccircle[angle=-90]{q2}{0.4cm}_{$0,1$} %und der rechts des Zustands
}
\end{pspicture}
\caption{automat}
\end{center}
\end{figure}

\begin{figure}[hbt]
\begin{center}
\begin{pspicture}(0, 0)(3.5, 5)
	\psset{unit=1cm}
	\psgrid
	\psplot{0}{2.5}{x 2 exp}
\end{pspicture}
\caption{Plot}
\end{center}
\end{figure}


\end{document}
