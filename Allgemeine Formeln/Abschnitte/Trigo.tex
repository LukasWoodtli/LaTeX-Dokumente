% Trigo.tex

\section{Trigonometrie}
\label{sec:Trigonometrie}

h: Hypotenuse\\
a: Ankathete\\
g: Gegenkathete
\[
sin(\alpha)=\frac{g}{h}
\]
\[
cos(\alpha)=\frac{a}{h}
\]
\[
tan(\alpha)=\frac{g}{a}
\]
\[
tan(\alpha)=\frac{sin(\alpha)}{cos(\alpha)}
\]
\[
sin^2(\alpha)+cos^2(\alpha)=1
\]


\section{Spezielle Winkel}
\label{sec:SpezielleWinkel}
Achtung! Nicht Ergebnisse aus den Tabellen zusammenz�hlen, um die Werte f�r die 
erw�nschten Zwischen-Winkel zu erhalten.\\
Sondern rechnen oder mit Einheitskreis herausfinden.

\begin{table*}[hb]
	\centering
		\begin{tabular}{|r|c|c|c|c|}
		\hline
			$\alpha$					&				$90^\circ$		&		$180^\circ$ 		&			$270^\circ$			&		 $360^\circ$\\
		\hline
		$ sin(\alpha)	$			&				1							&		0								&			-1							&			0		\\
		\hline
		$ cos(\alpha)	$			&				0							&		-1							&			0								&			1		\\
		\hline
		$ tan(\alpha)	$			&				-							&		0								&			-								&			0		\\
		\hline
			
		\end{tabular}
	\caption{Spezielle Winkel 1}
	\label{tab:SpezielleWinkel1}
\end{table*}


\begin{table*}[hb]
	\centering
		\begin{tabular}{|r|c|c|c|}
		\hline
			$\alpha$					&				$30^\circ$				&		$45^\circ$ 						&			$60^\circ$							\\
		\hline
		$ sin(\alpha)	$			&				$\frac{1}{2}$			&		$\frac{\sqrt{2}}{2}$	&			$\frac{\sqrt{3}}{2}$		\\
		\hline
		$ cos(\alpha)	$			&				$\frac{\sqrt{3}}{2}$ &$\frac{\sqrt{2}}{2}$	&			$\frac{1}{2}$		\\
		\hline
		$ tan(\alpha)	$			&				$\frac{\sqrt{3}}{3}$		&		1								&			$\sqrt{3}$\\
		\hline
			
		\end{tabular}
	\caption{Spezielle Winkel 2}
	\label{tab:SpezielleWinkel2}
\end{table*}



\section{Steigungswinkel}
\label{sec:Steigungswinkel}
m: Winkel in Prozent (\%) \\
$\alpha: \text{Winkel in Grad }(^\circ)$
\[
m=tan(\alpha)
\]
\[
\alpha=arctan(m)
\]
