\documentclass{article}
\usepackage{pst-plot,pstricks} % Zuerst die Pakete einbinden
% Hier eine sinnvolle und zur Positionierung von Beschriftungen
%�u�erst n�tzliche Zusatzfunktion
\newpsobject{showgrid}{psgrid}{subgriddiv=1,griddots=10,gridlabels=0pt}

\begin{document}

\psset{unit=1.0cm}         % Gr��e einer Einheit festlegen.
                           % Dar�ber l�sst sich die Grafik skalieren!
  \begin{pspicture}(-1.5,-1)(4,5.5) % Gr��e des dargestellten Bildes von

                                % x,y bis x,y
%  \showgrid                     % Raster
%  \psaxes{->}(0,0)(-3,0)(5,7)  % Achsen von x,y bis x,y Schnittpunkt
                                % der Achsen bei (0,0)
	\psline{<->}(0,5.5)(0,0)(3.5,0)
	\psline(0,0)(-1,0)
	\psline(3,0)(3,4)
	\psline(0,4)(3,4)
	\pswedge[]{5}{53.140}{98.130}
	\psline(-0.7071,0)(-0.7071,4.95)
	\psline(-0.7071,4.95)(0,4.95)
	\psarc[](0,0){0.5}{0}{53.140}
	\psarc[](0,0){0.8}{53.140}{98.140}
	
%  \rput(4.7,-0.4){$\mathbf{x}$}  % Beschriftung der x-Achse
% \rput(0.3,6.5){$\mathbf{y}$}  % Beschriftung der y-Achse

	\rput(0.6,0.3){$\alpha$}
	\rput(0.3,1){$45^{\circ}$}
	\rput(-0.7,-0.2){$x'$}
	\rput(3,-0.25){$x$}
	\rput(-0.2,2.5){$r$}
	\rput(1.6,2.5){$r$}
	\rput(-1,5){$A'$}
	\rput(3.2,4){$A$}
	\rput(-0.2,4){$y$}
	\rput(0.2,4.8){$y'$}
	%\rput(,){$$}


\end{pspicture}
\end{document}
