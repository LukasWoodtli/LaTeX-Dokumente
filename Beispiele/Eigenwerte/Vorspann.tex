% ----------------------------------------------------------------
% Unterlagen
% ----------------------------------------------------------------
\documentclass[11pt,oneside,titlepage,openany,a4paper]{book} % draft
\usepackage[T1]{fontenc}         % Aktivierung der ec-Zeichens�tze, NFSS
\usepackage{ngerman}             % f�r deutsche Trennregeln
\usepackage{inputenc}              % deutsche Umlaute
\usepackage{fancyhdr}            % Fuss- und Kopfzeile
%\usepackage{flafter}            % Gleitobjekte immer nach Text
\usepackage[intlimits]{amsmath}  % AMS-Latex
\usepackage{amssymb}             % AMS-Latex
\usepackage{dsfont,exscale}
\usepackage[thmmarks]{ntheorem}
\usepackage{pst-all}
\usepackage{pst-pdf}

%\usepackage{pstricks}
%\usepackage[usenames]{pstcol}
%\usepackage{pst-plot}
%\usepackage{pst-3dplot}
%\usepackage{pst-vue3d}
%\usepackage{pst-key}

 \pagestyle{fancy}
 \fancyhf{}
 \renewcommand{\headrulewidth}{0.4pt}
 \renewcommand{\footrulewidth}{0.4pt}
 \lhead{\sf \leftmark}
 \rhead{\thepage}
 \lfoot{\small Z�rcher Hochschule Winterthur}
 \rfoot{{\small Maz, \today} \ \texttt{\jobname.tex}}       %<--- file name %
 \setlength{\parskip}{3ex plus0.5ex minus0.5ex}
 \setlength{\parindent}{0em}
 \setlength{\headsep}{1.5cm}
 \setlength{\textwidth}{15cm}
 \setlength{\textheight}{22cm}
 \setlength{\headwidth}{15cm}
 \setlength{\headheight}{14pt}
 \setlength{\fboxsep}{0.3cm}
 \newfont{\titelschrift}{csssbx10 at 37pt}
 \newfont{\bbdingschrift}{bbding10 at 8pt}
 \definecolor{grau}{gray}{0.8}

\newcommand{\rahmen}[1]{\vspace*{1ex}\fcolorbox{black}{grau}{\parbox[b]{14.3cm}{\vspace*{-1.5ex}#1\vspace*{-0.5ex}}}\vspace*{2ex}}

\theoremstyle{break}
\theorembodyfont{\normalfont}
\setlength{\theorempreskipamount}{0.5ex plus0.5ex }
\setlength{\theorempostskipamount}{0ex}

\newtheorem{satz}{Satz}[chapter]
\newtheorem{definition}{Definition}[chapter]
\newtheorem{beispiel}{Beispiel}[chapter]
\newtheorem{aufgabe}{Aufgabe}

\theoremstyle{nonumberbreak}
\setlength{\theorempreskipamount}{2.5ex plus0.5ex
} \theoremsymbol{\ensuremath{\Box}}
\newtheorem{beweis}{Beweis}

\newcounter{liste123}
\newenvironment{list123}{\begin{list}
{\arabic{liste123})}{\usecounter{liste123}
\setlength{\parskip}{0ex plus0.5ex}
\setlength{\parsep}{0.5ex plus0.2ex minus0.1ex}
\setlength{\labelsep}{0.3cm}
\setlength{\leftmargin}{0.5cm}
\setlength{\labelwidth}{0.3cm}
\setlength{\partopsep}{0cm}
\setlength{\itemsep}{0.5ex}
}}
{\end{list}}

\newcommand{\folgerung}{{\bf Folgerung} \newline}
\newcommand{\bemerkung}{{\bf Bemerkung} \newline}
\newcommand{\refAbb}[1]{Abbildung \ref{#1}}

%\newcommand{\qed}{\hfill\mbox{\bbdingschrift\symbol{102}}}
\newcommand{\N}{\mathds N}
\newcommand{\Z}{\mathds Z}
\newcommand{\Q}{\mathds Q}
\newcommand{\R}{\mathds R}
\newcommand{\C}{\mathds C}

\newcommand{\vektor}[1]{
\begin{pmatrix}
  #1
\end{pmatrix}}
%  $\vektor{2\\3\\5\\7}$ Aufruf
