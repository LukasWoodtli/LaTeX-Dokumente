\documentclass[11pt,a4paper]{scrartcl}
\usepackage[ngerman]{babel}
\usepackage[latin1]{inputenc}
\usepackage[pdftex]{graphicx}
\usepackage[T1]{fontenc}
\usepackage{ae}
\usepackage{amsmath, amssymb}
\usepackage{wasysym, enumerate}
\usepackage{hyperref}

\parindent = 0.0in
%\pagestyle{empty}


\begin{document}
\title{Satz von Brennfleck/Kern\footnote{Martin Brennfleck und Stefan Kern waren Sch�ler meines Leistungskurses Mathematik, Abitur 1985.
Der hier aufgef�hrte Satz wurde w�hrend des Unterrichts von Martin Brennfleck formuliert und von Stefan Kern bewiesen. Aus Anlass des bald 20-j�hrigen Abitur-Jubil�ums der Genannten habe ich diese Seite gestaltet und m�chte damit meine Dankbarkeit f�r die hervorragende Kreativit�t und Mitwirkung bei der Unterrichtsgestaltung ausdr�cken.}}
\author{\href{mailto:hhorn@gmx.de}{Holger Horn}}
\maketitle
\thispagestyle{empty}
\underline{\textbf{Satz von Brennfleck}}:

Jede umkehrbare Funktion, deren Graph eine zusammenh�ngende Kurve ist, ist streng monoton.

$\Longleftrightarrow$

Wenn eine Funktion, deren Graph eine zusammenh�ngende Kurve ist, nicht streng monoton ist, dann ist sie nicht umkehrbar. (Negation)

\vspace{1cm}
\underline{\textbf{Beweis} der Negation} (Beweisidee von Stefan Kern):

Wegen der Voraussetzung f�r die Funktion $f$ sind nur zwei F�lle m�glich.

\begin{enumerate}[{Fall} 1:]
	\item 
	$f$ ist mindestens st�ckweise konstant, d.\,h. es gibt mindestens zwei Stellen $x_1$, $x_2$ ($x_1\neq x_2$) mit $f(x_1)=f(x_2)$
	
	$\Longrightarrow$ Umkehrrelation $\overline{f}$ von $f$ hat mindestens eine Stelle, an der Punkte mit verschiedenen Ordinaten ($y$-Werten) liegen.
	
	$\Longrightarrow$ $\overline{f}$ ist keine Funktion, d.\,h. $f$ ist nicht umkehrbar.
	
	\item
	Es gibt eine Stelle $x_0$, an der zwei Intervalle $I_1$, $I_2$ aneinander grenzen, �ber denen $f$ streng monoton ist: etwa �ber $I_1$ streng monoton fallend, �ber $I_2$ streng monoton wachsend.
	
	$\Longrightarrow$ In $x_0$ hat $f$ ein relatives Extremum (d.\,h. in einer gewissen Umgebung von $x_0$ sind alle Funktionswerte entweder kleiner oder gr��er als $f(x_0)).$
	
	$\Longrightarrow$ In einer Umgebung von $x_0$ gibt es Stellen $x_1$, $x_2$ mit $x_1<x_0$ und $x_2>x_0$, so dass $f(x_1)=f(x_2)$ ist.
	
	$\Longrightarrow$ (wie in Fall 1) $f$ ist nicht umkehrbar.
	 
\end{enumerate}

\end{document}
