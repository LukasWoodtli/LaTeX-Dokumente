
\documentclass[11pt,a4paper]{scrartcl}
\usepackage[latin1]{inputenc}  % Erlaube mehr als nur ASCII-Zeichen (z.B. Umlaute) im Text  
\usepackage[T1]{fontenc}       % Sorgt fuer richtige Trennung auch bei Umlauten im Wort
\usepackage{ngerman}           % bei alter Rechtschreibung hier german verwenden
\usepackage{ae}
%\usepackage{array}
\usepackage{tabularx}
\usepackage{amsmath,amssymb,fancybox}
%\usepackage[top=2.5cm,bottom=2.5cm,left=2.5cm,right=2.5cm]{geometry}
\usepackage[bottom=2.5cm]{geometry}
\usepackage{hyperref}

\parskip=1ex                    % zwischen Absaetzen eine Zeile frei lassen
\parindent=0pt                  % Absaetze nicht einruecken
%\pagestyle{empty}               % Keine Seitennummern oder Kopfzeilen

\usepackage[pdftex]{graphicx}
\DeclareGraphicsExtensions{.jpg}



\begin{document}
\title{Beweis f�r das Additionstheorem der Sinusfunktion $\mathbf{\left(\boldsymbol{\alpha, \beta} < 90^\circ\right)}$}
\author{\href{mailto:hhorn@gmx.de}{Holger Horn}}
\maketitle
\thispagestyle{empty}

\vspace{1cm}

Wegen $\sin(180^\circ-(\alpha + \beta))=\sin(\alpha + \beta)$ gilt im mittleren Dreieck (s. Zeichnung) nach dem Sinussatz:
$$\frac{\sin(\alpha + \beta)}{c}=\frac{\sin \delta}{1}$$
Da ferner $\sin \delta = h$ ist, folgt: $$\sin(\alpha + \beta)=hc$$

\centering \includegraphics[viewport= 231 344 365 498]{Bilder/AddTh.pdf}

\vspace{0.5cm}
F�r die Trapezfl�che ergibt sich nun einerseits:
$$A=(\cos \alpha + \cos \beta)\cdot \frac{1}{2}(\sin \alpha + \sin \beta)$$
und andererseits (als Summe dreier Dreiecksfl�chen):
$$A= \frac{1}{2}\sin \alpha \cos \alpha + \frac{1}{2}\sin \beta \cos \beta + \frac{1}{2}\sin (\alpha + \beta)$$
$$\Rightarrow \sin (\alpha + \beta) + \underline{\sin \alpha \cos \alpha} + \underline{\sin \beta \cos \beta} = \underline{\sin \alpha \cos \alpha} + \sin \beta \cos \alpha + \sin \alpha \cos \beta + \underline{\sin \beta \cos \beta}$$
$ \Rightarrow $ \Ovalbox{$ \sin (\alpha + \beta)= \sin \alpha \cos \beta + \sin \beta \cos \alpha $}
\end{document}