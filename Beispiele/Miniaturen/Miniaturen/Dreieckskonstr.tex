\documentclass[11pt,a4paper]{scrartcl}
\usepackage[ngerman]{babel}
\usepackage[applemac]{inputenc}
\usepackage{graphicx, url}
\DeclareGraphicsExtensions{.jpg}
\usepackage[T1]{fontenc}
\usepackage{ae}
\usepackage{amsmath, amssymb}
%\usepackage{wasysym}
\usepackage[top=1cm, left=2cm, right=2cm, bottom=2cm]{geometry}
\usepackage{hyperref}
\parindent = 0.0in
%\pagestyle{empty}


\begin{document}

\title{Konstruktion eines rechtwinkligen Dreiecks aus Kathete und ihr nicht anliegendem Hypotenusenabschnitt\footnote{L�sung gefunden im Internet unter der URL: \url{http://www.ijon.de/mathe/dreieck1/index.html}}}
\author{\href{mailto:hhorn@gmx.de}{Holger Horn}}
\maketitle

\thispagestyle{empty}

\textbf{Aufgabe:}\\

Konstruiere ein rechtwinkliges Dreieck aus einer Kathete und dem der anderen Kathete anliegenden Hypotenusenabschnitt.

Sei o.B.d.A. gegeben: $b, p$ (s. Schaufigur).

\begin{center}
\includegraphics[viewport=50 280 500 600, scale=0.8]{Bilder/Dreieck.pdf}
\end{center}

\textbf{Nachweis der Richtigkeit der Konstruktion:}\\

Nach dem Satz des Pythagoras und dem Kathetensatz gilt:
\begin{eqnarray*}
x^2&=&b^2+\frac{p^2}{4}\\
&=&(c-p)c+\frac{p^2}{4}\\
&=&\left(c-\frac{p}{2}\right)^2
\end{eqnarray*}
$$\Longrightarrow x= c -\frac{p}{2}\Longrightarrow x+\frac{p}{2}=c$$

\end{document}
