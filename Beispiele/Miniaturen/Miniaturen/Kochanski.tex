\documentclass[11pt,a4paper]{scrartcl}
\usepackage[ngerman]{babel}
\usepackage[latin1]{inputenc}
\usepackage[pdftex]{graphicx}
\usepackage[T1]{fontenc}
\usepackage{ae}
\usepackage{amsmath, amssymb}
%\usepackage{wasysym}
\usepackage{hyperref}

\parindent = 0.0in
%\pagestyle{empty}


\begin{document}
\title{N�herungskonstruktion f�r $\boldsymbol{\pi}$\\nach A. Kochanski}
\author{\href{mailto:hhorn@gmx.de}{Holger Horn}}
\maketitle
\thispagestyle{empty}
Konstruktionsbeschreibung:

Zeichne einen Kreis um $M$ mit Radius $r$ mit zwei zueinander senkrechten Durchmessern $\overline{AB}$ und $\overline{CF}$. Zeichne in $B$ die Tangente an den Kreis.
Zeichne �ber $\overline{CM}$ ein gleichseitiges Dreieck, dessen eine Ecke auf dem Kreisbogen $\stackrel{\frown}{BC}$ liegt. Verl�ngere die eine in $M$ beginnende Seite des Dreiecks, bis sie die Tangente in $D$ schneidet. Trage von $D$ aus auf der Tangente in Richtung von $B$ dreimal den Radius ab bis $E$.

Die Strecke $\overline{AE}=y$ ist N�herung f�r den halben Kreisumfang.
\begin{center}
\includegraphics[viewport= 152 574 442 791]{Bilder/Koch}
\end{center}
Zeige zun�chst, dass f�r $\overline{DM}=2x$ gilt: $\overline{DB}=x$.

Berechne dann $x$ und $y$. $\left(\text{Ergebnis:}\quad y=r\sqrt{13\frac{1}{3}-2\sqrt{3}}\right)$

Zeige, dass sich die o.a. Wurzel von $\pi$ erst in der f�nften Dezimale unterscheidet.

\end{document}
