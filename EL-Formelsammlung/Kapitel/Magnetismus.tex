% Magnetismus.tex

\section{Magnetismus}
\label{sec:Magnetismus}

\subsection{Die Durchflutung}
\label{sec:DieDurchflutung}
\begin{tabular}{l l l}
	\begin{minipage}[l]{5cm}
		\[
		\Theta = N \cdot I
		\]
		\[
		[\Theta] = A
		\]	
	\end{minipage}
	&
	\begin{minipage}[l]{5cm}
		$N$: Anzahl Wicklungen \\
		$I$: Stromst�rke
	\end{minipage}
\end{tabular}

\subsection{Die magn. Erregung (magn. Feldst�rke)}
\label{sec:DieMagnErregungMagnFeldst�rke}
\begin{tabular}{l l l}
	\begin{minipage}[l]{5cm}
		\[
		\vec{H} = \frac{N\cdot I}{\vec{l}} = \frac{\Theta}{\vec{l}}
		\]
		\[
		[\vec{H}] = \frac{A}{m}
		\]
	\end{minipage}
	&
	\begin{minipage}[l]{5cm}
		$\Theta$: Durchflutung \\
		$\vec{l}$: L�nge der Feldlinie (Richtung: Rechte-Hand-Regel)
	\end{minipage}
\end{tabular}

\subsection{Die magn. Spannung}
\label{sec:DieMagnSpannung}
\begin{tabular}{l l l}
	\begin{minipage}[l]{5cm}
		\[
		V_m = \int \limits _S \vec{H} \ ds
		\]	
		\[
		[V_m]= \frac{A}{m} \cdot m = A
		\]
	\end{minipage}
\end{tabular}

\subsection{Die magn. Flussdichte (magn. Induktion)}
\label{sec:DieMagnFlussdichteMagnInduktion}
\begin{tabular}{l l l}
	\begin{minipage}[l]{5cm}
		\[
		\vec{B} = \frac{\vec{F}}{I \cdot \vec{l}} \quad \text{f�r} \vec{B} \bot \vec{l}
		\]
		\[
		\vec{F} = I \cdot (\vec{l} \times \vec{B}) 
		\]
		\[
		[\vec{B}] = \frac{N}{Am} = \frac{Vs}{m^2} = T \; \text{(Tesla)}
		\]
	\end{minipage}
	&
	\begin{minipage}[l]{5cm}
		$\vec{B}$: magn. Flussdichte \\
		$\vec{F}$: Lorenz-Kraft \\
		$\vec{l}$: L�nge des Leiters in Richtung von $I$
	\end{minipage}
\end{tabular}

\subsection{Der magn. Fluss}
\label{sec:DerMagnFluss}
\begin{tabular}{l l l}
	\begin{minipage}[l]{5cm}
		\[
		\Phi = \int \limits _A \vec{B} \ d\vec{A}
		\]
		\[
		[\Phi] = Vs = Wb \; \text{(Weber)}
		\]
	\end{minipage}
\end{tabular}

\subsection{Die Permeabilit�t}
\label{sec:DiePermeabilit�t}
\begin{tabular}{l l l}
	\begin{minipage}[l]{5cm}
		\[
		\vec{B} = \mu \cdot \vec{H}
		\]
		\[
		\mu = \mu_0 \cdot \mu_r
		\]	
	\end{minipage}
\end{tabular}
\begin{itemize}
	\item $\mu$ ist in Metallen nicht konstant (Magnetisierungskennlinie)
\end{itemize}

\subsection{Das Durchflutungsgesetz}
\label{sec:DasDurchflutungsgesetz}
\begin{tabular}{l l l}
	\begin{minipage}[l]{5cm}
		\[
		\Theta = \oint \limits _{S} \vec{H} \ d\vec{s}
		\]
	\end{minipage}
	\begin{minipage}[l]{5cm}
		$\Theta$: gesammte umschlossene Durchflutung
	\end{minipage}
\end{tabular}

\subsection{Der magn. Widerstand}
\label{sec:DerMagnWiderstand}
\begin{tabular}{l l l}
	\begin{minipage}[l]{5cm}
		\[
		R_m = \frac{1}{\mu} \cdot \frac{l}{A}
		\]
		\[
		\Lambda = \frac{1}{R_m}
		\]
		\[
		[R_m] = \frac{A}{Vs}
		\]
	\end{minipage}
\end{tabular}
\begin{itemize}
	\item Schaltungen mit magnetischen Widers�nden sind analog zu elektrischen Widers�nden zu berechnen.
\end{itemize}


\subsection{Das ohmsche Gesetz des magn. Kreises}
\label{sec:DasOhmscheGesetzDesMagnKreises}
\begin{tabular}{l l l}
	\begin{minipage}[l]{5cm}
		\[
		\Phi = \frac{\Theta}{R_m}
		\]
	\end{minipage}
	&
	\begin{minipage}[l]{5cm}
		\[
		\text{analog: } I = \frac{U}{R}
		\]
	\end{minipage}
\end{tabular}