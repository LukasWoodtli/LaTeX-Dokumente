% Induktion.tex
\section{Induktion}
\label{sec:Induktion}

\subsection{Die induzierte Spannung}
\label{sec:DieInduzierteSpannung}

\begin{tabular}{l l}
	\begin{minipage}[l]{5cm}
		\[
			U=\int \limits^2_1 \vec{E} \ d \vec{l}
			=-\int \limits^2_1 (\vec{v} \times \vec{B}) \ d \vec{l}
		\]
		\[
			U=B\cdot l\cdot v \quad \text{f�r} \; \vec{B} \bot \vec{v}
		\]
	\end{minipage}
\end{tabular}

\subsection{Das Induktionsgesetz}
\label{sec:DasInduktionsgesetz}

\begin{tabular}{l l l}
	\begin{minipage}[l]{5cm}
		\[
		U=N\cdot \frac{d\Phi}{dt}
		\]
	\end{minipage}
\end{tabular}

\subsection{Lenz'sche Regel}
\label{sec:LenzScheRegel}
Die induzierte Spannung ist stets so gerichtet, dass ein durch sie hervorgerufener Strom der Ursache ihrer Entstehung entgegenwirkt.


\subsection{Die Induktivit�t (Spule)}
\label{sec:DieInduktivit�tSpule}
\begin{tabular}{l l l}
	\begin{minipage}[l]{5cm}
	\[
		L= \frac{N^2}{\sum{R_m}}
	\]
	\[
		[L]= \frac{Vs}{A}=H
	\]
	\[
		U_L = L \cdot \frac{di(t)}{dt}
	\]
\end{minipage}
\end{tabular}

Der Strom durch die Induktivit�t kann nicht springen.

\subsection{Die differentielle Permeabilit�t}
\label{sec:DieDifferentiellePermeabilit�t}

\begin{tabular}{l l l}
	\begin{minipage}[l]{5cm}
		\[
			\mu_{dif} = \frac{dB}{dH}
		\]
		\[
			L=N^2 \cdot \mu_{dif}\cdot \frac{A}{l}
	\]
	\end{minipage}
	&
	\begin{minipage}[l]{5cm}
		$\mu_{dif}$: Differentielle Permeabilit�t (Steigung der Hysteresekurve)
	\end{minipage}
\end{tabular}

\begin{itemize}
		\item Bei gegebener Geometrie variiert L ausschliesslich mit der Steigung der Hysteresekurve. Bei kleinen H-Werten werden wir einsehr grosses L erhalten. Im S�ttigungsbereich kann L um mehrere zehnerpotenzen zusammenfallen.
	
	\item Die Formel $L=N^2 \cdot \mu_{dif}\cdot \frac{A}{l}$ gilt nur f�r magnetische Kreise aus \emph{einem} Material.
	
	\item Es ist ersichtlich, dass die Induktivit�t einer gegebene Spule mit ferromagnetischem Kern um ein Vielfaches gr�sser ist, als die selbe Spule ohne Kern.

\end{itemize}

\subsection{Grundschaltungen mit Iduktivit�ten}
\label{sec:GrundschaltungenMitIduktivit�ten}

\subsubsection{Serieschaltung:}
\label{sec:Serieschaltung}

\begin{tabular}{l l l}
	\begin{minipage}[l]{5cm}
		\[
			L_{Tot} = \sum{L_i}
		\]
	\end{minipage}
\end{tabular}

\subsubsection{Parallelschaltung}
\label{sec:Parallelschaltung}
\begin{tabular}{l l l}
	\begin{minipage}[l]{5cm}
		\[
			\frac{1}{L_{Tot}} = \sum{\frac{1}{L_i}}
		\]
	\end{minipage}
\end{tabular}

\begin{itemize}
	\item 
		Bei Schaltungen mit Induktivit�ten wird vorausgestzt, dass sich die Magnetischen Felder der einzelnen Spulen nicht beeinflussen.
	\item
		Die Formel f�r die Parallelschaltung gilt nur f�r ideale Induktivit�ten, also solche ohne Innenwiderstand und ohne parasit�re Kapazit�ten. Bei hohen Frequenzen darf nicht so gerechnet werden.
		\item
			Das zusammenschalten von Induktivit�ten hat in der Praxis fast keine Bedeutung, da Spulen meisst "`massgefertigt"' werden.
\end{itemize}

\section{Gegeninduktivit�t}
\label{sec:Gegeninduktivit�t}

\subsection{Gegeniduktion}
\label{sec:Gegeniduktion}

\begin{tabular}{l l l}
	\begin{minipage}[l]{5cm}
		\[
		L_{ij} = L_{ji} = k \cdot \sqrt{L_i \cdot L_j}
		\]	
	\end{minipage}
	&
	\begin{minipage}[l]{5cm}
		$L_n$: Induktivit�ten	\\
		$k$: Kopplungsfaktor
	\end{minipage}
\end{tabular}

	
\begin{itemize}
	\item
		Ist $k\approx 1$ spricht man von einer \emph{engen Kopplung} zweier Induktivit�ten. Ist $k$ jedoch nahe bei 0, so nennt man das eine \emph{lose Kopplung}.
		\item
			Der Kopplungsfaktor ist nur von den elektromechanischen Eigenschaften abh�ngig. Er ist weder von der Leistung noch von der Frequenz abh�ngig (gilt nur f�r tiefe bis mittlere Frequenzen).
			\item
				Gegeninduktion ist eine einfache Form der drahtlosen Energie�bertragung. Beispiele: Transformator (enge Kopplung), Signal�bertragung, Antennen, Sender (lose Kopplung)
			\item
				�ber diese Beziehung lassen sich auch "`unfreiwillig gekoppelte"' magnetische Kreise beschreiben.
\end{itemize}

\subsection{Transformator (ideal)}
\label{sec:TransformatorIdeal}

\begin{tabular}{l l l}
	\begin{minipage}[l]{5cm}
		\[
			\frac{u_1}{u_2} = \frac{N_1}{N_2}
		\]
		\[
			\frac{i_1}{i_2} = \frac{N_2}{N_1}
		\]
	\end{minipage}
\end{tabular}

\begin{itemize}
	\item 
		Es lassen sich nur Wechselgr�ssen �bertragen.
\end{itemize}

\section{Energie und Kr�fte im magnetischen Feld}
\label{sec:EnergieUndKr�fteImMagnetischenFeld}

\subsection{Der Energieinhalt der Induktivit�t}
\label{sec:DerEnergieinhaltDerInduktivit�t}
\begin{tabular}{l l l}
	\begin{minipage}[l]{5cm}
		\[
		W(i) = L \cdot \int \limits _{i_1}^{i_2} i\ di
		\]
		\[
		W(i) = \frac{L}{2} \cdot I^2 \quad \text{f�r} \ i_1 = 0
		\]
	\end{minipage}
\end{tabular}

\subsection{Die Energiedichte}
\label{sec:DieEnergiedichte}
\begin{tabular}{l l l}
	\begin{minipage}[l]{5cm}
		\[
		w = \int \limits _{B_1} ^{B_2} \vec{H} \ d\vec{B}
		\]
		\[
		w = \frac{1}{\mu} \int \limits _{B_1} ^{B_2} \vec{B} \ d\vec{B}
		\]
		\[
		[w]= \frac{J}{m^3}
		\]
	\end{minipage}
	&
	\begin{minipage}[l]{5cm}
		$w$: Energiedichte ($w= \frac{dW}{dV}$)
	\end{minipage}
\end{tabular}
\begin{itemize}
	\item Bei \emph{nicht konstanter} Permeabilit�t muss man das Integral f�r die Energiedichte in der Regel graphisch l�sen.
\end{itemize}

\subsection{Kr�fte an Grenzfl�chen}
\label{sec:Kr�fteAnGrenzfl�chen}
\begin{tabular}{l l l}
	\begin{minipage}[l]{5cm}
		\[
		F = \frac{B^2}{2\mu_0} \cdot A
		\]
		\[
		F = \frac{\Phi^2}{2\mu_0 \cdot A}
		\]
	\end{minipage}
	&
	\begin{minipage}[l]{5cm}
		$B$: magnetische Flussdichte im Luftspalt \\
		$A$: Fl�che des Luftspaltes
	\end{minipage}
\end{tabular}