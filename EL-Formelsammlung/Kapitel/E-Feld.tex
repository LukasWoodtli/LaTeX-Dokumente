% E-Feld.tex

\section{E-Feld}
\label{sec:EFeld}

\subsection{Kr�fte zwischen Ladungen}
\label{sec:Kr�fteZwischenLadungen}
\begin{tabular}{l l l}
	\begin{minipage}[l]{5cm}
		
		\[
		F=\frac{1}{4\pi\epsilon_0}\cdot \frac{\left|Q_1 \cdot Q_2\right|}{r^2}
		\]
	\end{minipage}
	&
	\begin{minipage}[l]{5cm}
		$F$: Kraft zwischen den Ladungen  \\
		$\epsilon_0$: Dielektrizit�tskonstante \\
		$r$: Abstand der Ladungen
	\end{minipage}
\end{tabular}

\subsection{Das elektrische Feld}
\label{sec:DasElektrischeFeld}

\[
\vec{E}= \frac{\vec{F}}{Q}
\]

\[
[\vec{E}]= \frac{N}{As} =\frac{V}{m}
\]

\subsection{Die elektrische Verschiebungsdichte}
\label{sec:DieElektrischeVerschiebungsdichte}

\[
\vec{D}=\epsilon_r \cdot \epsilon_0 \cdot \vec{E}
\]

\[
\vec{D}= \frac{\Psi}{A}=\frac{Q}{A}
\]

\[
[D]= \frac{As}{m^2}
\]

\subsection{Der elektrische Fluss}
\label{sec:DerElektrischeFluss}

\[
\Psi = Q
\]

\subsection{Spannung und Leistung im elektrischen Feld}
\label{sec:SpannungUndLeistungImElektrischenFeld}

\[
	U_{12}= \frac{W_{12}}{q} =\int \limits^2_1{\vec{E}(s) \ ds}
\]

\subsection{Satz von Gauss}
\label{sec:SatzVonGauss}

\[
	Q=\oint \limits_A\vec{D} \ d\vec{A}
\]

\subsection{Die Kapazit�t}
\label{sec:DieKapazit�t}

\[
	U = Q \cdot \int\frac{1}{\epsilon \cdot A} \ ds
\]
	\[
	Q = C \cdot U
\]

\subsection{Energie des elektrostatischen Feldes}
\label{sec:EnergieDesElektrostatischenFeldes}

\[
W= \frac{1}{2}C \cdot U^2 = \frac{1}{2} \frac{Q^2}{C}
\]

