\section{SI Einheitensystem}
\label{sec:SIEinheitensystem}


\subsection{SI Einheiten}
\label{sec:SIEinheiten}

Siehe Tabelle \ref{tab:SIEinheiten} Seite \pageref{tab:SIEinheiten}.

\begin{table*}[hb]
	\centering
		\begin{tabular}{|c|l|c|}
		\hline
		Symbol			&				Name								&		Einheit						\\
		\hline
		$ I	$				&		Stromst�rke							&		$ A $									\\
		$ U	$				&		Spannung								&		$ V	$								\\
		$ P	$				&		Leistung								&		$ W	$								\\
		$ W / E	$		&		Arbeit / Energie				&		$ J	$								\\
		$ Q	$				&		Ladung									&		$ C	$								\\
		$ R	$				&		Widerstand							&		$ \Omega $					\\
		$ G	$				&		Leitwert								&		$ S $									\\
		$ J $				&		Stromdichte							&		$ {A}/{m^2} $		\\
		$\rho$			&		spezifischer Widerstand	&		$ {\Omega\;m^2}/{m}	$	\\
		$ t $				&		Zeit										&		$ s $											\\
		\hline
			
		\end{tabular}
	\caption{SI Einheiten}
	\label{tab:SIEinheiten}
\end{table*}


\subsection{Vors�tze von dezimalen Vielfachen (SI)}
\label{sec:Vors�tzeVonDezimalenVielfachenSI}

Siehe Tabelle \ref{tbl:SIVorsatzzeichen} Seite \pageref{tbl:SIVorsatzzeichen}.

\begin{table} [hb]
	\centering
	\begin{tabular}{|r|c|l|}
	\hline
	Vorsatz	&	Zeichen	&	Bedeutung	\\
	\hline
	Tera		&	T				&	$10^{12}$	\\
	Giga		&	G				&	$10^9$	\\
	Mega		&	M				&	$10^6$	\\
	\hline
	Kilo		&	k				&	$10^3$	\\
	Hekto		&	h				&	$10^2$	\\
	Deka		&	da			&	$10^1$	\\	
	\hline
	Dezi		&	d				&	$10^{-1}$	\\
	Zenti		&	c				&	$10^{-2}$	\\
	Milli		&	m				&	$10^{-3}$	\\
	\hline
	Mikro		&	$\mu$		&$10^{-6}$	\\
	Nano		&	n				&	$10^{-9}$	\\
	Piko		&	p				&	$10^{-12}$\\
	\hline
	Femto		&	f				&	$10^{-15}$\\
	Atto		&	a				&	$10^{-18}$\\
	\hline
	\end{tabular}
	\caption{SI Vorsatzzeichen}
	\label{tbl:SIVorsatzzeichen}
\end{table}
	

\section{Physikalische Konstanten}
\label{sec:PhysikalischeKonstanten}
\paragraph{Elementarladung}
\label{sec:Elementarladung}
\[
e=1,602\cdot10^{-19}\; C
\]
