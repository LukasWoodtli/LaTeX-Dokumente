% Die G�tter

\chapter{Die G�tter}
\label{sec:DieG�tter}
Die G�tter spielen vom Anfang der Zeit eine grosse Rolle in der griechischen Mythologie. Sie intervenieren immer wieder in menschlichen Angelegenheiten und sind in vielen Geschischten verwickelt. 

\section{Athene}
\label{sec:Athene}


\section{Chaos}
\label{sec:Chaos}
Chaos ist der ungeordnete Urzustand des Universums aus dem die ersten G�tter entstanden sind. Diese ersten G�tter heissen: Tartaros(Dunkles Gebiet unter der Erde), Nyx (Nacht), Erebos (Dunkelheit), Eros (Liebe) und Gaia(Erde).

\section{Echidna}
\label{sec:Echidna}
Echidna war die Mutter zalreicher Ungeheuer.  �ber ihre Abstammung gibt es verschiedene Erz�hlungen.
Sie sei die Tochter des Phorkys, eines Alten des Meeres, und der Keto, eines Meeresungeheuers oder aber der Kallirhoe und des Chrysaor, der dem Leib seiner Mutter Medusa entsprungen war, als Perseus diese enthauptete oder auch der Gaia und des Tartaros. Mit Thyphon zeugte sie den zweik�pfigen Orthos, den dreik�pfigen Kerberos und die Hydra. Mit ihrem Sohn Orthos zeugte sie dann die Chimaira, die Sphinx, den Nemeischen L�wen und die Phaia, ein ungeheures, w�tendes Schwein.

\section{Erebos}
\label{sec:Erebos}
Erebos ist eines der Kinder des Caos. Er verk�rpert die Finsternis der Erdentiefe. Mit Nyx (Nacht) hat Erebos viele Kinder, darunter Traum, Schlaf, Verderben, Alter, Tod, Zwietracht, �rger, Elend, Entsagung, die Nemesis, die Moiren und die Hesperieden aber auch die Freude, die Freundschaft und das Mitleid. Zudem gilt die Styx auch als Tochter der beiden, die jedoch auch Okeanine die �lteste Tochter von Okeanos und der Tethys ist.

\section{Eros}
\label{sec:Eros}
Eros ist der Gott der Liebe. Einerseits soll er mit den ersten G�ttern aus dem Chaos entstanden sein andererseits gilt er als Sohn der Aphrodite und des Ares. Es stellt jedoch ein Paradoxon dar, dass er Sohn des Ares ist, da Eros als der Vater von Uranos gilt, welcher wiederum der Urgrossvater von Ares ist.

\section{Gaia}
\label{sec:Gaia}
Die Mutter Erde wird Gaia genannt. Sie war die Mutter von Okeanos, der Titanen und Kyklopen. Diese Kinder hatte sie mit Uranos gezeugt, der auch ein Sohn von ihr war. Sp�ter hatte Gaia noch weitere Kinder mit Pontos, darunter Nereus, Keto, Phorkys.

\section{Helios}
\label{sec:Helios}

\section{Hypnos}
\label{sec:Hypnos}
Hypnos ist der Gott des Schlafes. Sein Zwillingsbruder ist Thanatos der Tod. Sei sind beide aus Nyx entstanden. Hypnos wird mit seinem Zwillingsbruder Thanatos h�ufig als die so genannte Ildefonso-Gruppe mit Schlaf und Tod dargestellt. �blicherweise wird Hypnos mit Schetterlingsfl�geln auf den Schl�fen dargestellt.

\section{Nyx}
\label{sec:Nyx}
Nyx ist die G�ttin der Nacht und der Finsternis. Sie ist aus dem Chaos mit den anderen ersten G�ttern entstanden. 

Es gibt jedoch �berlieferungen, nach denen Nyx schon von Anfang an dagewesen ist. Sie sei vom Wind Aithir befruchtet worden und legte ein silbernes Ei in den Schoss der Dunkelheit. Aus dem Ei ist dann Eros geschl�pft. In dem Ei jedoch war oben auch eine Leere, das Chaos drin. Unten im Ei war die Grosse Mutter Gaia drin. 

\section{Pontos}
\label{sec:Pontos}
Pontos ist eine Meergottheit. Er ist der Sohn von Gaia und Aether. Er zeugte mit Gaia Nereus, Thaumas, Phorkys, Keto und Eurybia. Mit Thalassa zeugte er die Telchinen.

\section{Thanatos}
\label{sec:Thanatos}
Thanatos ist der Gott des Todes beziehungsweise der personifizierte Tod. Er ist der Zwillingsbruder vom Schlaf Hypnos. Ausserdem hat er noch die Geschwister: der Eris, der Ker, des Momos und Nemesis. Seine Mutter ist die Nyx.

\section{Uranos}
\label{sec:Uranos}
Uranos ist der Gott des Himmelgew�lbes. Mit Gaia hat er die Titanen, die Kyklopen und die Hekatoncheiren gezeugt. Ausserdem sind aus ihm Aphrodite, die Erinyen (Furien), die Giganten und die Meliaden entstanden. Er ist einer der ersten G�tter und der erstgeborene von Gaia. Mit ihm kahm das m�nnliche Element in die Welt.